% Options for packages loaded elsewhere
\PassOptionsToPackage{unicode}{hyperref}
\PassOptionsToPackage{hyphens}{url}
%
\documentclass[
]{article}
\usepackage{amsmath,amssymb}
\usepackage{iftex}
\ifPDFTeX
  \usepackage[T1]{fontenc}
  \usepackage[utf8]{inputenc}
  \usepackage{textcomp} % provide euro and other symbols
\else % if luatex or xetex
  \usepackage{unicode-math} % this also loads fontspec
  \defaultfontfeatures{Scale=MatchLowercase}
  \defaultfontfeatures[\rmfamily]{Ligatures=TeX,Scale=1}
\fi
\usepackage{lmodern}
\ifPDFTeX\else
  % xetex/luatex font selection
\fi
% Use upquote if available, for straight quotes in verbatim environments
\IfFileExists{upquote.sty}{\usepackage{upquote}}{}
\IfFileExists{microtype.sty}{% use microtype if available
  \usepackage[]{microtype}
  \UseMicrotypeSet[protrusion]{basicmath} % disable protrusion for tt fonts
}{}
\makeatletter
\@ifundefined{KOMAClassName}{% if non-KOMA class
  \IfFileExists{parskip.sty}{%
    \usepackage{parskip}
  }{% else
    \setlength{\parindent}{0pt}
    \setlength{\parskip}{6pt plus 2pt minus 1pt}}
}{% if KOMA class
  \KOMAoptions{parskip=half}}
\makeatother
\usepackage{xcolor}
\usepackage[margin=1in]{geometry}
\usepackage{color}
\usepackage{fancyvrb}
\newcommand{\VerbBar}{|}
\newcommand{\VERB}{\Verb[commandchars=\\\{\}]}
\DefineVerbatimEnvironment{Highlighting}{Verbatim}{commandchars=\\\{\}}
% Add ',fontsize=\small' for more characters per line
\usepackage{framed}
\definecolor{shadecolor}{RGB}{248,248,248}
\newenvironment{Shaded}{\begin{snugshade}}{\end{snugshade}}
\newcommand{\AlertTok}[1]{\textcolor[rgb]{0.94,0.16,0.16}{#1}}
\newcommand{\AnnotationTok}[1]{\textcolor[rgb]{0.56,0.35,0.01}{\textbf{\textit{#1}}}}
\newcommand{\AttributeTok}[1]{\textcolor[rgb]{0.13,0.29,0.53}{#1}}
\newcommand{\BaseNTok}[1]{\textcolor[rgb]{0.00,0.00,0.81}{#1}}
\newcommand{\BuiltInTok}[1]{#1}
\newcommand{\CharTok}[1]{\textcolor[rgb]{0.31,0.60,0.02}{#1}}
\newcommand{\CommentTok}[1]{\textcolor[rgb]{0.56,0.35,0.01}{\textit{#1}}}
\newcommand{\CommentVarTok}[1]{\textcolor[rgb]{0.56,0.35,0.01}{\textbf{\textit{#1}}}}
\newcommand{\ConstantTok}[1]{\textcolor[rgb]{0.56,0.35,0.01}{#1}}
\newcommand{\ControlFlowTok}[1]{\textcolor[rgb]{0.13,0.29,0.53}{\textbf{#1}}}
\newcommand{\DataTypeTok}[1]{\textcolor[rgb]{0.13,0.29,0.53}{#1}}
\newcommand{\DecValTok}[1]{\textcolor[rgb]{0.00,0.00,0.81}{#1}}
\newcommand{\DocumentationTok}[1]{\textcolor[rgb]{0.56,0.35,0.01}{\textbf{\textit{#1}}}}
\newcommand{\ErrorTok}[1]{\textcolor[rgb]{0.64,0.00,0.00}{\textbf{#1}}}
\newcommand{\ExtensionTok}[1]{#1}
\newcommand{\FloatTok}[1]{\textcolor[rgb]{0.00,0.00,0.81}{#1}}
\newcommand{\FunctionTok}[1]{\textcolor[rgb]{0.13,0.29,0.53}{\textbf{#1}}}
\newcommand{\ImportTok}[1]{#1}
\newcommand{\InformationTok}[1]{\textcolor[rgb]{0.56,0.35,0.01}{\textbf{\textit{#1}}}}
\newcommand{\KeywordTok}[1]{\textcolor[rgb]{0.13,0.29,0.53}{\textbf{#1}}}
\newcommand{\NormalTok}[1]{#1}
\newcommand{\OperatorTok}[1]{\textcolor[rgb]{0.81,0.36,0.00}{\textbf{#1}}}
\newcommand{\OtherTok}[1]{\textcolor[rgb]{0.56,0.35,0.01}{#1}}
\newcommand{\PreprocessorTok}[1]{\textcolor[rgb]{0.56,0.35,0.01}{\textit{#1}}}
\newcommand{\RegionMarkerTok}[1]{#1}
\newcommand{\SpecialCharTok}[1]{\textcolor[rgb]{0.81,0.36,0.00}{\textbf{#1}}}
\newcommand{\SpecialStringTok}[1]{\textcolor[rgb]{0.31,0.60,0.02}{#1}}
\newcommand{\StringTok}[1]{\textcolor[rgb]{0.31,0.60,0.02}{#1}}
\newcommand{\VariableTok}[1]{\textcolor[rgb]{0.00,0.00,0.00}{#1}}
\newcommand{\VerbatimStringTok}[1]{\textcolor[rgb]{0.31,0.60,0.02}{#1}}
\newcommand{\WarningTok}[1]{\textcolor[rgb]{0.56,0.35,0.01}{\textbf{\textit{#1}}}}
\usepackage{graphicx}
\makeatletter
\def\maxwidth{\ifdim\Gin@nat@width>\linewidth\linewidth\else\Gin@nat@width\fi}
\def\maxheight{\ifdim\Gin@nat@height>\textheight\textheight\else\Gin@nat@height\fi}
\makeatother
% Scale images if necessary, so that they will not overflow the page
% margins by default, and it is still possible to overwrite the defaults
% using explicit options in \includegraphics[width, height, ...]{}
\setkeys{Gin}{width=\maxwidth,height=\maxheight,keepaspectratio}
% Set default figure placement to htbp
\makeatletter
\def\fps@figure{htbp}
\makeatother
\setlength{\emergencystretch}{3em} % prevent overfull lines
\providecommand{\tightlist}{%
  \setlength{\itemsep}{0pt}\setlength{\parskip}{0pt}}
\setcounter{secnumdepth}{-\maxdimen} % remove section numbering
\ifLuaTeX
  \usepackage{selnolig}  % disable illegal ligatures
\fi
\usepackage{bookmark}
\IfFileExists{xurl.sty}{\usepackage{xurl}}{} % add URL line breaks if available
\urlstyle{same}
\hypersetup{
  pdftitle={Rasch-model},
  pdfauthor={Merlin},
  hidelinks,
  pdfcreator={LaTeX via pandoc}}

\title{Rasch-model}
\author{Merlin}
\date{2024-10-09}

\begin{document}
\maketitle

\section{Load Packages}\label{load-packages}

\begin{Shaded}
\begin{Highlighting}[]
\CommentTok{\# Required package for matrix operations}
\FunctionTok{library}\NormalTok{(MASS)  }\CommentTok{\# For the ginv() function (generalized inverse)}
\end{Highlighting}
\end{Shaded}

\section{The data simulation}\label{the-data-simulation}

\begin{Shaded}
\begin{Highlighting}[]
\NormalTok{generate\_irt\_data }\OtherTok{\textless{}{-}} \ControlFlowTok{function}\NormalTok{(}\AttributeTok{num\_items =} \DecValTok{3}\NormalTok{, }\AttributeTok{num\_persons =} \DecValTok{1000}\NormalTok{, }\AttributeTok{seed =} \DecValTok{124}\NormalTok{) \{}
  
  \FunctionTok{set.seed}\NormalTok{(seed) }\CommentTok{\# Set seed for reproducibility}
  
  \CommentTok{\# Step 1: Define Item Parameters}
\NormalTok{  beta\_true }\OtherTok{\textless{}{-}} \FunctionTok{runif}\NormalTok{(num\_items, }\SpecialCharTok{{-}}\DecValTok{2}\NormalTok{, }\DecValTok{2}\NormalTok{) }\CommentTok{\# uniformly distributed item difficulties around 0 logits}
\NormalTok{  beta\_true }\OtherTok{\textless{}{-}} \FunctionTok{scale}\NormalTok{(beta\_true, }\AttributeTok{scale =} \ConstantTok{FALSE}\NormalTok{) }\CommentTok{\# Center the betas because they are relative}
  
  \CommentTok{\# Step 2: Define Person Parameters with fixed mean and SD}
\NormalTok{  person\_ability }\OtherTok{\textless{}{-}} \FunctionTok{rnorm}\NormalTok{(num\_persons, }\AttributeTok{mean =} \SpecialCharTok{{-}}\DecValTok{1}\NormalTok{, }\AttributeTok{sd =} \DecValTok{1}\NormalTok{) }\CommentTok{\# normally distributed person abilities}
\NormalTok{  person\_ability }\OtherTok{\textless{}{-}} \FunctionTok{scale}\NormalTok{(person\_ability, }\AttributeTok{scale =} \ConstantTok{FALSE}\NormalTok{) }\CommentTok{\# Center the thetas because they are relative}
  
  \CommentTok{\# Step 4: Simulate Responses}
  \CommentTok{\# Create an empty response matrix to store responses for each person{-}item pair}
\NormalTok{  response\_matrix }\OtherTok{\textless{}{-}} \FunctionTok{matrix}\NormalTok{(}\DecValTok{0}\NormalTok{, }\AttributeTok{nrow =}\NormalTok{ num\_persons, }\AttributeTok{ncol =}\NormalTok{ num\_items)}
  
  \ControlFlowTok{for}\NormalTok{ (i }\ControlFlowTok{in} \DecValTok{1}\SpecialCharTok{:}\NormalTok{num\_persons) \{}
    \ControlFlowTok{for}\NormalTok{ (j }\ControlFlowTok{in} \DecValTok{1}\SpecialCharTok{:}\NormalTok{num\_items) \{}
      \CommentTok{\# Step 4A: Generate a random number U from uniform [0,1]}
\NormalTok{      U }\OtherTok{\textless{}{-}} \FunctionTok{runif}\NormalTok{(}\DecValTok{1}\NormalTok{)}
      
      \CommentTok{\# Step 4B: Compute probability of failure}
\NormalTok{      prob\_failure }\OtherTok{\textless{}{-}} \DecValTok{1} \SpecialCharTok{/}\NormalTok{ (}\DecValTok{1} \SpecialCharTok{+} \FunctionTok{exp}\NormalTok{(person\_ability[i] }\SpecialCharTok{{-}}\NormalTok{ beta\_true[j]))}
      
      \CommentTok{\# Step 4C: Check if U \textgreater{} Probability of failure}
      \ControlFlowTok{if}\NormalTok{ (U }\SpecialCharTok{\textgreater{}}\NormalTok{ prob\_failure) \{}
\NormalTok{        response\_matrix[i, j] }\OtherTok{\textless{}{-}} \DecValTok{1} \CommentTok{\# Correct response}
\NormalTok{      \} }\ControlFlowTok{else}\NormalTok{ \{}
\NormalTok{        response\_matrix[i, j] }\OtherTok{\textless{}{-}} \DecValTok{0} \CommentTok{\# Incorrect response}
\NormalTok{      \}}
\NormalTok{    \}}
\NormalTok{  \}}
  
  \CommentTok{\# Return both the response matrix and the true beta values}
  \FunctionTok{return}\NormalTok{(}\FunctionTok{list}\NormalTok{(}\AttributeTok{response\_matrix =}\NormalTok{ response\_matrix, }\AttributeTok{beta\_true =}\NormalTok{ beta\_true))}
\NormalTok{\}}
\end{Highlighting}
\end{Shaded}

\subsection{Generate Example Data}\label{generate-example-data}

\begin{Shaded}
\begin{Highlighting}[]
\CommentTok{\# Generate example data with 5 items and 1000 persons}
\NormalTok{result }\OtherTok{\textless{}{-}} \FunctionTok{generate\_irt\_data}\NormalTok{(}\AttributeTok{num\_items =} \DecValTok{5}\NormalTok{, }\AttributeTok{num\_persons =} \DecValTok{1000}\NormalTok{)}

\NormalTok{response\_matrix }\OtherTok{\textless{}{-}}\NormalTok{ result}\SpecialCharTok{$}\NormalTok{response\_matrix}
\NormalTok{beta\_true }\OtherTok{\textless{}{-}}\NormalTok{ result}\SpecialCharTok{$}\NormalTok{beta\_true}
\end{Highlighting}
\end{Shaded}

\section{The Estimation}\label{the-estimation}

\subsection{Sum Algorithm for the Rasch
Model}\label{sum-algorithm-for-the-rasch-model}

Initialize Matrix:

\begin{verbatim}
Creates a matrix \text{results} of size (J+1)×(J+1), where J is the number of items:
results=[11…100…0⋮⋮⋱⋮00…0]
\end{verbatim}

Compute Item Difficulty Transformation:

\begin{verbatim}
Sets bi=e−βi for each item i, where βi is the item difficulty:
bi=e−βi
\end{verbatim}

Fill Matrix Using Recurrence Relation:

\begin{verbatim}
Loops over each item i and score k to fill in the \text{results} matrix with:
results[k+1,i+1]=bi⋅results[k,i]+results[k+1,i]
\end{verbatim}

Return Probability of Observed Score:

\begin{verbatim}
Returns the probability of the given score, \text{results}[ \text{score} + 1, J + 1], from the last column of \text{results} for the observed score row.
\end{verbatim}

\begin{Shaded}
\begin{Highlighting}[]
\CommentTok{\# Define the sum algorithm for the Rasch model}
\NormalTok{sum\_algorithm\_prob }\OtherTok{\textless{}{-}} \ControlFlowTok{function}\NormalTok{(beta, score) \{}
\NormalTok{  results }\OtherTok{\textless{}{-}} \FunctionTok{matrix}\NormalTok{(}\DecValTok{0}\NormalTok{, }\AttributeTok{nrow =} \FunctionTok{length}\NormalTok{(beta) }\SpecialCharTok{+} \DecValTok{1}\NormalTok{, }\AttributeTok{ncol =} \FunctionTok{length}\NormalTok{(beta) }\SpecialCharTok{+} \DecValTok{1}\NormalTok{) }\CommentTok{\# make a results matrix filled with ones with J+1 columns and rows, where J is the number of items. }
\NormalTok{  results[}\DecValTok{1}\NormalTok{, ] }\OtherTok{\textless{}{-}} \DecValTok{1} \CommentTok{\# set values in the first row to 1}
\NormalTok{  bi }\OtherTok{\textless{}{-}} \FunctionTok{exp}\NormalTok{(}\SpecialCharTok{{-}}\NormalTok{beta) }\CommentTok{\# set b\_i as exp({-}Beta\_i)}
  
  \ControlFlowTok{for}\NormalTok{ (i }\ControlFlowTok{in} \DecValTok{1}\SpecialCharTok{:}\FunctionTok{length}\NormalTok{(beta)) \{}
    \ControlFlowTok{for}\NormalTok{ (k }\ControlFlowTok{in} \DecValTok{1}\SpecialCharTok{:}\FunctionTok{length}\NormalTok{(beta)) \{}
\NormalTok{      results[k }\SpecialCharTok{+} \DecValTok{1}\NormalTok{, i }\SpecialCharTok{+} \DecValTok{1}\NormalTok{] }\OtherTok{\textless{}{-}}\NormalTok{ bi[i] }\SpecialCharTok{*}\NormalTok{ results[k, i] }\SpecialCharTok{+}\NormalTok{ results[k }\SpecialCharTok{+} \DecValTok{1}\NormalTok{, i] }\CommentTok{\# fill in the results matrix with the actual results. }
\NormalTok{    \}}
\NormalTok{  \}}
  \FunctionTok{return}\NormalTok{(results[score }\SpecialCharTok{+} \DecValTok{1}\NormalTok{, }\FunctionTok{length}\NormalTok{(beta) }\SpecialCharTok{+} \DecValTok{1}\NormalTok{])  }\CommentTok{\# +1 to account for the possibility of 0 score}
\NormalTok{\}}
\end{Highlighting}
\end{Shaded}

\begin{Shaded}
\begin{Highlighting}[]
\CommentTok{\# Define the log{-}likelihood for all beta values using the sum algorithm}
\NormalTok{logLik\_rasch }\OtherTok{\textless{}{-}} \ControlFlowTok{function}\NormalTok{(beta, response\_matrix) \{}
\NormalTok{  N }\OtherTok{\textless{}{-}} \FunctionTok{nrow}\NormalTok{(response\_matrix)  }\CommentTok{\# Number of persons}
\NormalTok{  J }\OtherTok{\textless{}{-}} \FunctionTok{ncol}\NormalTok{(response\_matrix)  }\CommentTok{\# Number of items}
  
\NormalTok{  S\_i }\OtherTok{\textless{}{-}} \FunctionTok{rowSums}\NormalTok{(response\_matrix)  }\CommentTok{\# Total correct responses per person}
\NormalTok{  log\_likelihood }\OtherTok{\textless{}{-}} \DecValTok{0} \CommentTok{\# attribute a log likelihood of 0 for now}
  
  \CommentTok{\# Compute the likelihood for each person}
  \ControlFlowTok{for}\NormalTok{ (i }\ControlFlowTok{in} \DecValTok{1}\SpecialCharTok{:}\NormalTok{N) \{}
\NormalTok{    observed\_score }\OtherTok{\textless{}{-}}\NormalTok{ S\_i[i]  }\CommentTok{\# Total score for person i}
\NormalTok{    prob\_score }\OtherTok{\textless{}{-}} \FunctionTok{sum\_algorithm\_prob}\NormalTok{(beta, observed\_score)  }\CommentTok{\# Probability of the score}
\NormalTok{    log\_likelihood }\OtherTok{\textless{}{-}}\NormalTok{ log\_likelihood }\SpecialCharTok{+} \FunctionTok{sum}\NormalTok{(response\_matrix[i, ] }\SpecialCharTok{*}\NormalTok{ (}\SpecialCharTok{{-}}\NormalTok{beta)) }\SpecialCharTok{{-}} \FunctionTok{log}\NormalTok{(prob\_score) }\CommentTok{\# using {-}beta in the log{-}likelihood}
\NormalTok{  \}}
  
  \FunctionTok{return}\NormalTok{(log\_likelihood)  }\CommentTok{\# Return the log{-}likelihood}
\NormalTok{\}}
\end{Highlighting}
\end{Shaded}

\begin{Shaded}
\begin{Highlighting}[]
\CommentTok{\# {-}{-}{-}{-}{-}{-}{-}{-}{-}{-}{-}{-}{-}{-}{-}{-}{-}{-}{-}{-}{-}{-}{-}{-}{-}{-}{-}{-}{-}{-}{-}{-}{-}{-}{-}{-}{-}{-}{-}{-}{-}{-}{-}{-}{-}{-}{-}{-}{-}{-}{-}{-}{-}{-}{-}{-}{-}{-}{-}{-}{-}{-}{-}{-}{-}{-}{-}{-}{-}{-}{-}{-}{-}{-}{-}{-}{-}{-}}

\CommentTok{\# Define the gradient (first derivative) of the log{-}likelihood using the sum algorithm}
\NormalTok{grad\_rasch }\OtherTok{\textless{}{-}} \ControlFlowTok{function}\NormalTok{(beta, response\_matrix) \{}
\NormalTok{  N }\OtherTok{\textless{}{-}} \FunctionTok{nrow}\NormalTok{(response\_matrix) }\CommentTok{\# Number of persons}
\NormalTok{  J }\OtherTok{\textless{}{-}} \FunctionTok{ncol}\NormalTok{(response\_matrix) }\CommentTok{\# Number of items}
  
\NormalTok{  S\_i }\OtherTok{\textless{}{-}} \FunctionTok{rowSums}\NormalTok{(response\_matrix)  }\CommentTok{\# Total correct responses per person}
\NormalTok{  gradient }\OtherTok{\textless{}{-}} \FunctionTok{numeric}\NormalTok{(J) }\CommentTok{\# create vector of 0s of length J }
  
\NormalTok{  bi }\OtherTok{\textless{}{-}} \FunctionTok{exp}\NormalTok{(}\SpecialCharTok{{-}}\NormalTok{beta)  }\CommentTok{\# Compute exp({-}beta) for all items}
  
  \CommentTok{\# Compute the gradient for each beta\_j}
  \ControlFlowTok{for}\NormalTok{ (j }\ControlFlowTok{in} \DecValTok{1}\SpecialCharTok{:}\NormalTok{J) \{}
    \CommentTok{\# print(paste("j i", j))}
\NormalTok{    grad\_sum }\OtherTok{\textless{}{-}} \DecValTok{0} 
    \ControlFlowTok{for}\NormalTok{ (i }\ControlFlowTok{in} \DecValTok{1}\SpecialCharTok{:}\NormalTok{N) \{}
          \CommentTok{\# print(paste("i i", i))}
\NormalTok{      observed\_score }\OtherTok{\textless{}{-}}\NormalTok{ S\_i[i]  }\CommentTok{\# Total score for person i}
\NormalTok{      prob\_score }\OtherTok{\textless{}{-}} \FunctionTok{sum\_algorithm\_prob}\NormalTok{(beta, observed\_score)  }\CommentTok{\# Probability of the total score}
\NormalTok{      beta\_without\_j }\OtherTok{\textless{}{-}}\NormalTok{ beta[}\SpecialCharTok{{-}}\NormalTok{j] }\CommentTok{\# Remove current beta for the derivative following beta}
\NormalTok{      prev\_prob }\OtherTok{\textless{}{-}} \FunctionTok{sum\_algorithm\_prob}\NormalTok{(beta\_without\_j, observed\_score }\SpecialCharTok{{-}} \DecValTok{1}\NormalTok{)}
      
      \ControlFlowTok{if}\NormalTok{ (observed\_score }\SpecialCharTok{\textgreater{}} \DecValTok{0}\NormalTok{) \{  }\CommentTok{\# Failsafe: Only calculate if gradient is defined}
\NormalTok{        grad\_sum }\OtherTok{\textless{}{-}}\NormalTok{ grad\_sum }\SpecialCharTok{+}\NormalTok{ (}\SpecialCharTok{{-}}\NormalTok{response\_matrix[i, j]) }\SpecialCharTok{+}\NormalTok{ (bi[j] }\SpecialCharTok{*}\NormalTok{ prev\_prob) }\SpecialCharTok{/}\NormalTok{ prob\_score  }\CommentTok{\# Adjust with sum algorithm}
\NormalTok{      \}}
      \CommentTok{\# print(grad\_sum)}
\NormalTok{    \}}
\NormalTok{    gradient[j] }\OtherTok{\textless{}{-}}\NormalTok{ grad\_sum}
\NormalTok{  \}}
  
  \FunctionTok{return}\NormalTok{(gradient)  }\CommentTok{\# Return the gradient}
\NormalTok{\}}
\end{Highlighting}
\end{Shaded}

\begin{Shaded}
\begin{Highlighting}[]
\CommentTok{\# Define the Hessian (second derivative) for the Newton{-}Raphson update, using sum algorithm}
\NormalTok{hessian\_rasch }\OtherTok{\textless{}{-}} \ControlFlowTok{function}\NormalTok{(beta, response\_matrix) \{}
\NormalTok{  N }\OtherTok{\textless{}{-}} \FunctionTok{nrow}\NormalTok{(response\_matrix) }\CommentTok{\# Number of persons}
\NormalTok{  J }\OtherTok{\textless{}{-}} \FunctionTok{ncol}\NormalTok{(response\_matrix) }\CommentTok{\# Number of items}

\NormalTok{  S\_i }\OtherTok{\textless{}{-}} \FunctionTok{rowSums}\NormalTok{(response\_matrix) }\CommentTok{\# Total correct responses per person}
\NormalTok{  hessian }\OtherTok{\textless{}{-}} \FunctionTok{matrix}\NormalTok{(}\DecValTok{0}\NormalTok{, J, J)  }\CommentTok{\# Initialize Hessian matrix}
  
\NormalTok{  bi }\OtherTok{\textless{}{-}} \FunctionTok{exp}\NormalTok{(}\SpecialCharTok{{-}}\NormalTok{beta) }\CommentTok{\# Compute exp({-}beta) for all items}
  
  \CommentTok{\# Compute the Hessian for each beta\_j and beta\_k (cross{-}terms)}
  \ControlFlowTok{for}\NormalTok{ (j }\ControlFlowTok{in} \DecValTok{1}\SpecialCharTok{:}\NormalTok{J) \{}
    \ControlFlowTok{for}\NormalTok{ (k }\ControlFlowTok{in} \DecValTok{1}\SpecialCharTok{:}\NormalTok{J) \{}
\NormalTok{      hess\_sum }\OtherTok{\textless{}{-}} \DecValTok{0}
      \ControlFlowTok{for}\NormalTok{ (i }\ControlFlowTok{in} \DecValTok{1}\SpecialCharTok{:}\NormalTok{N) \{}
\NormalTok{        observed\_score }\OtherTok{\textless{}{-}}\NormalTok{ S\_i[i]  }\CommentTok{\# Total score for person i}
\NormalTok{        prob\_score }\OtherTok{\textless{}{-}} \FunctionTok{sum\_algorithm\_prob}\NormalTok{(beta, observed\_score)  }\CommentTok{\# Probability of the total score}
        

          \ControlFlowTok{if}\NormalTok{ (j }\SpecialCharTok{==}\NormalTok{ k) \{}
            \ControlFlowTok{if}\NormalTok{ (observed\_score }\SpecialCharTok{\textgreater{}} \DecValTok{0}\NormalTok{) \{  }\CommentTok{\# Failsafe: Only calculate if the Hessian is defined}
\NormalTok{              beta\_without\_j }\OtherTok{\textless{}{-}}\NormalTok{ beta[}\SpecialCharTok{{-}}\NormalTok{j] }\CommentTok{\# Remove current beta j = k for the derivative following beta j = k}
\NormalTok{              prev\_prob }\OtherTok{\textless{}{-}} \FunctionTok{sum\_algorithm\_prob}\NormalTok{(beta\_without\_j, observed\_score }\SpecialCharTok{{-}} \DecValTok{1}\NormalTok{)}
\NormalTok{              hess\_sum }\OtherTok{\textless{}{-}}\NormalTok{ hess\_sum }\SpecialCharTok{{-}}\NormalTok{ (bi[j] }\SpecialCharTok{*}\NormalTok{ prev\_prob) }\SpecialCharTok{/}\NormalTok{ prob\_score }\SpecialCharTok{+}\NormalTok{ ((bi[j] }\SpecialCharTok{*}\NormalTok{ prev\_prob) }\SpecialCharTok{/}\NormalTok{ prob\_score)}\SpecialCharTok{\^{}}\DecValTok{2}
\NormalTok{            \}}
\NormalTok{          \} }\ControlFlowTok{else}\NormalTok{ \{}
            \ControlFlowTok{if}\NormalTok{ (observed\_score }\SpecialCharTok{\textgreater{}} \DecValTok{1}\NormalTok{) \{  }\CommentTok{\# Failsafe: Only calculate if the Hessian is defined}
\NormalTok{              beta\_without\_j\_k }\OtherTok{\textless{}{-}}\NormalTok{ beta[}\SpecialCharTok{{-}}\FunctionTok{c}\NormalTok{(j, k)] }\CommentTok{\# Remove current betas j and k for the derivative following beta j and k}
\NormalTok{              prev\_prob\_j\_k }\OtherTok{\textless{}{-}} \FunctionTok{sum\_algorithm\_prob}\NormalTok{(beta\_without\_j\_k, observed\_score }\SpecialCharTok{{-}} \DecValTok{2}\NormalTok{)}
\NormalTok{              prev\_prob\_j }\OtherTok{\textless{}{-}} \FunctionTok{sum\_algorithm\_prob}\NormalTok{(beta[}\SpecialCharTok{{-}}\NormalTok{j], observed\_score }\SpecialCharTok{{-}} \DecValTok{1}\NormalTok{)}
\NormalTok{              prev\_prob\_k }\OtherTok{\textless{}{-}} \FunctionTok{sum\_algorithm\_prob}\NormalTok{(beta[}\SpecialCharTok{{-}}\NormalTok{k], observed\_score }\SpecialCharTok{{-}} \DecValTok{1}\NormalTok{)}
\NormalTok{              hess\_sum }\OtherTok{\textless{}{-}}\NormalTok{ hess\_sum }\SpecialCharTok{{-}}\NormalTok{ (bi[j] }\SpecialCharTok{*}\NormalTok{ bi[k] }\SpecialCharTok{*}\NormalTok{ prev\_prob\_j\_k) }\SpecialCharTok{/}\NormalTok{ prob\_score }\SpecialCharTok{+}\NormalTok{ (bi[j] }\SpecialCharTok{*}\NormalTok{ bi[k] }\SpecialCharTok{*}\NormalTok{ prev\_prob\_j }\SpecialCharTok{*}\NormalTok{ prev\_prob\_k) }\SpecialCharTok{/}\NormalTok{ prob\_score}\SpecialCharTok{\^{}}\DecValTok{2}
\NormalTok{          \}}
\NormalTok{        \}}
\NormalTok{      \}}
\NormalTok{      hessian[j, k] }\OtherTok{\textless{}{-}}\NormalTok{ hess\_sum}
\NormalTok{    \}}
\NormalTok{  \}}
  
  \FunctionTok{return}\NormalTok{(hessian)  }\CommentTok{\# Return the Hessian matrix}
\NormalTok{\}}
\end{Highlighting}
\end{Shaded}

\begin{Shaded}
\begin{Highlighting}[]
\CommentTok{\# Multivariate Newton{-}Raphson implementation}
\NormalTok{multivariate\_newton\_raphson }\OtherTok{\textless{}{-}} \ControlFlowTok{function}\NormalTok{(beta\_init, response\_matrix, }\AttributeTok{tol =} \FloatTok{1e{-}6}\NormalTok{, }\AttributeTok{max\_iter =} \DecValTok{1000}\NormalTok{, }\AttributeTok{save\_iterations =} \ConstantTok{FALSE}\NormalTok{) \{}
\NormalTok{  beta }\OtherTok{\textless{}{-}}\NormalTok{ beta\_init}
\NormalTok{  J }\OtherTok{\textless{}{-}} \FunctionTok{length}\NormalTok{(beta\_init)  }\CommentTok{\# Number of items}
\NormalTok{  beta\_history }\OtherTok{\textless{}{-}} \FunctionTok{matrix}\NormalTok{(}\ConstantTok{NA}\NormalTok{, }\AttributeTok{nrow =}\NormalTok{ max\_iter, }\AttributeTok{ncol =}\NormalTok{ J)  }\CommentTok{\# Initialize beta history matrix}
  
  \ControlFlowTok{for}\NormalTok{ (iter }\ControlFlowTok{in} \DecValTok{1}\SpecialCharTok{:}\NormalTok{max\_iter) \{}
    \FunctionTok{print}\NormalTok{(iter) }\CommentTok{\# To track progress}
\NormalTok{    gradient }\OtherTok{\textless{}{-}} \FunctionTok{grad\_rasch}\NormalTok{(beta, response\_matrix)}
\NormalTok{    hessian }\OtherTok{\textless{}{-}} \FunctionTok{hessian\_rasch}\NormalTok{(beta, response\_matrix)}
    
    \CommentTok{\# Update rule: beta\_new = beta\_old {-} H\_inv * grad}
\NormalTok{    beta\_new }\OtherTok{\textless{}{-}}\NormalTok{ beta }\SpecialCharTok{{-}} \FunctionTok{ginv}\NormalTok{(hessian) }\SpecialCharTok{\%*\%}\NormalTok{ gradient }\CommentTok{\# Use gradient and hessian derived with respect to beta}
    \FunctionTok{print}\NormalTok{(}\FunctionTok{max}\NormalTok{(}\FunctionTok{abs}\NormalTok{(beta\_new }\SpecialCharTok{{-}}\NormalTok{ beta)))  }\CommentTok{\# Check the max change for convergence}
    
    \CommentTok{\# Save the current beta values if save\_iterations is TRUE}
    \ControlFlowTok{if}\NormalTok{ (save\_iterations) \{}
\NormalTok{      beta\_history[iter, ] }\OtherTok{\textless{}{-}}\NormalTok{ beta  }\CommentTok{\# Store the current beta in the matrix}
\NormalTok{    \}}
    
    \CommentTok{\# Check for convergence}
    \ControlFlowTok{if}\NormalTok{ (}\FunctionTok{max}\NormalTok{(}\FunctionTok{abs}\NormalTok{(beta\_new }\SpecialCharTok{{-}}\NormalTok{ beta)) }\SpecialCharTok{\textless{}}\NormalTok{ tol) \{}
      \FunctionTok{cat}\NormalTok{(}\StringTok{"Converged after"}\NormalTok{, iter, }\StringTok{"iterations}\SpecialCharTok{\textbackslash{}n}\StringTok{"}\NormalTok{)}
      \CommentTok{\# Trim the beta\_history matrix to the actual number of iterations}
      \ControlFlowTok{if}\NormalTok{ (save\_iterations) \{}
\NormalTok{        beta\_history }\OtherTok{\textless{}{-}}\NormalTok{ beta\_history[}\DecValTok{1}\SpecialCharTok{:}\NormalTok{iter, ]}
\NormalTok{      \}}
      \FunctionTok{return}\NormalTok{(}\FunctionTok{list}\NormalTok{(}\AttributeTok{beta =}\NormalTok{ beta\_new, }\AttributeTok{beta\_history =}\NormalTok{ beta\_history))}
\NormalTok{    \}}
    
    \CommentTok{\# Update beta}
\NormalTok{    beta }\OtherTok{\textless{}{-}}\NormalTok{ beta\_new}
\NormalTok{  \}}
  
  \FunctionTok{cat}\NormalTok{(}\StringTok{"Did not converge within the maximum number of iterations}\SpecialCharTok{\textbackslash{}n}\StringTok{"}\NormalTok{)}
  \FunctionTok{return}\NormalTok{(}\FunctionTok{list}\NormalTok{(}\AttributeTok{beta =}\NormalTok{ beta, }\AttributeTok{beta\_history =}\NormalTok{ beta\_history))}
\NormalTok{\}}

\CommentTok{\# Initial guess for beta}
\NormalTok{beta\_init }\OtherTok{\textless{}{-}} \FunctionTok{rep}\NormalTok{(}\DecValTok{0}\NormalTok{, }\FunctionTok{ncol}\NormalTok{(response\_matrix))}

\CommentTok{\# Run the multivariate Newton{-}Raphson algorithm}
\NormalTok{beta\_output }\OtherTok{\textless{}{-}} \FunctionTok{multivariate\_newton\_raphson}\NormalTok{(beta\_init, response\_matrix, }\AttributeTok{save\_iterations =} \ConstantTok{TRUE}\NormalTok{)}
\end{Highlighting}
\end{Shaded}

\begin{verbatim}
## [1] 1
## [1] 0.9785134
## [1] 2
## [1] 0.06033565
## [1] 3
## [1] 0.005738605
## [1] 4
## [1] 0.0007773908
## [1] 5
## [1] 0.0001041692
## [1] 6
## [1] 1.402361e-05
## [1] 7
## [1] 1.888007e-06
## [1] 8
## [1] 2.542175e-07
## Converged after 8 iterations
\end{verbatim}

\begin{Shaded}
\begin{Highlighting}[]
\CommentTok{\# Print the final estimated beta values}
\NormalTok{beta\_estimated }\OtherTok{\textless{}{-}}\NormalTok{ beta\_output}\SpecialCharTok{$}\NormalTok{beta}
\FunctionTok{print}\NormalTok{(beta\_estimated)}
\end{Highlighting}
\end{Shaded}

\begin{verbatim}
##            [,1]
## [1,] -1.0333843
## [2,]  0.3659261
## [3,]  0.8271829
## [4,]  0.3659261
## [5,] -0.4426789
\end{verbatim}

\begin{Shaded}
\begin{Highlighting}[]
\CommentTok{\# Compare the estimated beta with the true beta}
\NormalTok{comparison }\OtherTok{\textless{}{-}} \FunctionTok{data.frame}\NormalTok{(}
  \AttributeTok{Item =} \DecValTok{1}\SpecialCharTok{:}\FunctionTok{length}\NormalTok{(beta\_true),}
  \AttributeTok{True\_Beta =}\NormalTok{ beta\_true,}
  \AttributeTok{Estimated\_Beta =}\NormalTok{ beta\_estimated}
\NormalTok{)}

\CommentTok{\# Display the comparison}
\FunctionTok{print}\NormalTok{(comparison)}
\end{Highlighting}
\end{Shaded}

\begin{verbatim}
##   Item  True_Beta Estimated_Beta
## 1    1 -0.9692984     -1.0333843
## 2    2  0.3338185      0.3659261
## 3    3  0.7597799      0.8271829
## 4    4  0.2861693      0.3659261
## 5    5 -0.4104693     -0.4426789
\end{verbatim}

The comparison with the rasch model

\begin{Shaded}
\begin{Highlighting}[]
\FunctionTok{library}\NormalTok{(ltm)}
\end{Highlighting}
\end{Shaded}

\begin{verbatim}
## Lade nötiges Paket: msm
\end{verbatim}

\begin{verbatim}
## Lade nötiges Paket: polycor
\end{verbatim}

\begin{Shaded}
\begin{Highlighting}[]
\NormalTok{rasch\_model }\OtherTok{\textless{}{-}} \FunctionTok{rasch}\NormalTok{(response\_matrix)}
\NormalTok{beta\_estimates\_ltm }\OtherTok{\textless{}{-}} \FunctionTok{coef}\NormalTok{(rasch\_model)}
\NormalTok{beta\_estimates\_ltm}
\end{Highlighting}
\end{Shaded}

\begin{verbatim}
##          Dffclt   Dscrmn
## [1,] -1.0284330 1.057578
## [2,]  0.2905511 1.057578
## [3,]  0.7284704 1.057578
## [4,]  0.2905514 1.057578
## [5,] -0.4755501 1.057578
\end{verbatim}

Graph that shows convergence:

\begin{Shaded}
\begin{Highlighting}[]
\FunctionTok{library}\NormalTok{(ggplot2) }\CommentTok{\# For plotting}
\FunctionTok{library}\NormalTok{(reshape2) }\CommentTok{\# For melt}

\CommentTok{\# Since we already saved the beta history, we can plot it:}
\NormalTok{beta\_history }\OtherTok{\textless{}{-}}\NormalTok{ beta\_output}\SpecialCharTok{$}\NormalTok{beta\_history}

\CommentTok{\# Convert beta\_history to a data frame for plotting}
\NormalTok{beta\_history\_df }\OtherTok{\textless{}{-}} \FunctionTok{as.data.frame}\NormalTok{(beta\_history)}
\NormalTok{beta\_history\_df}\SpecialCharTok{$}\NormalTok{Iteration }\OtherTok{\textless{}{-}} \DecValTok{1}\SpecialCharTok{:}\FunctionTok{nrow}\NormalTok{(beta\_history\_df)}
\NormalTok{beta\_history\_long }\OtherTok{\textless{}{-}}\NormalTok{ reshape2}\SpecialCharTok{::}\FunctionTok{melt}\NormalTok{(beta\_history\_df, }\AttributeTok{id.vars =} \StringTok{"Iteration"}\NormalTok{, }\AttributeTok{variable.name =} \StringTok{"Item"}\NormalTok{, }\AttributeTok{value.name =} \StringTok{"Beta"}\NormalTok{)}

\CommentTok{\# Rename the Item variable for clearer legend labels}
\NormalTok{beta\_history\_long}\SpecialCharTok{$}\NormalTok{Item }\OtherTok{\textless{}{-}} \FunctionTok{factor}\NormalTok{(beta\_history\_long}\SpecialCharTok{$}\NormalTok{Item, }\AttributeTok{labels =} \FunctionTok{paste0}\NormalTok{(}\StringTok{"Item "}\NormalTok{, }\DecValTok{1}\SpecialCharTok{:}\FunctionTok{ncol}\NormalTok{(beta\_history)))}

\CommentTok{\# Plot convergence of each beta value with customized legend labels}
\FunctionTok{ggplot}\NormalTok{(beta\_history\_long, }\FunctionTok{aes}\NormalTok{(}\AttributeTok{x =}\NormalTok{ Iteration, }\AttributeTok{y =}\NormalTok{ Beta, }\AttributeTok{color =}\NormalTok{ Item)) }\SpecialCharTok{+}
  \FunctionTok{geom\_line}\NormalTok{() }\SpecialCharTok{+}
  \FunctionTok{labs}\NormalTok{(}\AttributeTok{title =} \StringTok{"Convergence of beta values across iterations"}\NormalTok{, }\AttributeTok{x =} \StringTok{"Iteration"}\NormalTok{, }\AttributeTok{y =} \StringTok{"Beta Estimate"}\NormalTok{, }\AttributeTok{color =} \StringTok{"Item"}\NormalTok{) }\SpecialCharTok{+}
  \FunctionTok{theme\_minimal}\NormalTok{()}
\end{Highlighting}
\end{Shaded}

\includegraphics{Newton-Raphson-in-progress-with-comments_files/figure-latex/unnamed-chunk-10-1.pdf}

\end{document}
